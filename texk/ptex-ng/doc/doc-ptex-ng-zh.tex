%!TeX Program=pLaTeX-ng
\special{pdf: mapfile fandol-embeded-scheme1.map}
\documentclass[dvipdfmx]{beamer}
%\pdfminorversion=5
%\pdfcompresslevel=5
\usepackage{hologo}
%\usepackage{tgpagella}
\usetheme{EastLansing}
\usefonttheme{serif}
\setbeamerfont{frametitle}{series=\bfseries}
%%%%
\ybaselineshift.5pt
\pdfpagewidth\paperwidth
\pdfpageheight\paperheight
%%%%
\newcommand{\pTeX}{p\kern-.1667em\TeX}
\newcommand{\ptexng}{\pTeX-ng}
\newcommand{\yandy}{Y\kern-.21em{\tiny\&}\kern-.12emY}
\newcommand{\JTeX}{\leavevmode\hbox{\lower.5ex\hbox{J}\kern-.18em\TeX}}
\newcommand{\PUTeX}{\leavevmode\hbox{P\kern-.18em\lower.5ex\hbox{U}\kern-.18em\TeX}}
%%%%
\begin{document}
\title{\bf \TeX 与汉字处理:过去,现在,将来}
\author{马起园}
\date{2015年—2016年}
\begin{frame}
\maketitle
\begin{center}
\includegraphics[bb=0 0 255 197, width=9em]{tuna-logo.png}
\end{center}
\end{frame}
%
\parskip.5zw
%
\begin{frame}[fragile]
\frametitle{\ptexng 是什么}
\ptexng 的设计目标是成为是下一代汉字处理的标准\pTeX。在底层引擎上支持:汉字处理、
禁则处理、间距处理、直排、OpenType处理,多语言处理、UTF-8编码支持。对于排版输出
的基本规则,\ptexng 遵循日本JIS标准X 4051及中国大陆、中国台湾的行业惯用法。

\pTeX 是日本{\font\t=ascii10 \t ASCII}公司开发的汉字处理引擎。在2008年和2010年
经历合并及收购之后,\pTeX 的开发已经停止,后续维护由\TeX\ Users Group开发
团队进行,但最近几年并无新功能加入。

\ptexng 的开发最早可以追溯到2012年春节假期,最早在Lua\TeX 上进行试验性质的修改。
在2013年一度中断开发,转向读代码和读文档,2014年年初最先使用Common \TeX 进行扩展,
后转向Y\&Y \TeX 进行二次开发。\ptexng 第一版在2014年10月发布,使用GPL第二版许可。
\end{frame}
%
\begin{frame}[fragile]
\frametitle{\ptexng 的开发概要}

\ptexng 的前身是Y\&Y公司的一个用C语言实现的带内存管理的\TeX。
早期源码是使用{\texttt{web2c}}进行转换得到,不具可读性和可扩展性。
\ptexng 的源码建立于2014年重写代码的基础之上。

\ptexng 的的汉字处理相关代码来源于\pTeX;Unicode编码处理代码来源于u\pTeX。
对于\hologo{eTeX}的支持来源于eu\pTeX。对于OpenType的处理,来源于\texttt{libm17n}。

\ptexng 只支持PDF文件输出,相关代码来自\texttt{dvipdfmx},没有使用任何来自
{\textsc{pdf}\TeX}的代码,所以大量PDF输出相关的primitive并不能在\ptexng
中使用。
\end{frame}
%
\begin{frame}[fragile]
\frametitle{\TeX 扩展:面向汉字处理}
%http://www.weblio.jp/wkpja/content/TeX_TeX+%E3%81%AB%E3%82%88%E3%82%8B%E7%B5%84%E7%89%88%E3%81%AE%E4%BD%9C%E6%A5%AD%E5%B7%A5%E7%A8%8B
%http://tug.org/TUGboat/tb08-2/tb18saito.pdf
%https://www.tug.org/TUGboat/Articles/tb11-3/tb29hamano.pdf
%http://ja.wikipedia.org/wiki/Publishing_TeX
%http://oku.edu.mie-u.ac.jp/~okumura/texwiki/?TeX%E3%81%AE%E6%AD%B4%E5%8F%B2
%http://www.cs.pu.edu.tw/~tsay/putex/
\begin{itemize}
\item Fujita \TeX ,由Hiroshi Fujita开发,基于\TeX78(SAIL语言)。由于现在
没有更多的材料,所以这个扩展已经无从考证了。
\item NTT \JTeX ,由斋藤康己(Yasuki Saito)开发,起于1988年,在1994年得到NTT的
ASTEM Software Culture Award。Unix版本由千叶大学的樱井贵文(Takafumi Sakurai)移植。该扩展在Debian
以及W32\TeX 中均可安装,但未进入\TeX{} Live。\JTeX 的开发思路,是在字体处理上做了扩展,这样在处理
起来需要的补丁量不大。\JTeX 支持类似magin kerning的burasage。
\item ASCII p\TeX ,ASCII公司的大野俊治(Shunji Ohno)和仓泽良一(Ryoichi
Kurasawa)等开发,起于1987年(昭和62年)。p\TeX 中的p代表的是publishing的意思。p\TeX 的设计相对于
\JTeX 来说,逻辑更为复杂,比如支持直行排版。
\item \PUTeX ,台湾静宜大学资管系蔡奇伟开发,起于1997年,止于2004年。最后版本为
\PUTeX4。\PUTeX 提供了很多primitive来实现了Virtual Font功能,需要使用\texttt{cidpdfmx}生成PDF。
\end{itemize}
\end{frame}
%
\begin{frame}[fragile]
\frametitle{扩展的难点}
\begin{itemize}
\item 字符集编码形式
\item 字体操作
\end{itemize}
\end{frame}
%
\begin{frame}[fragile]
\frametitle{扫描器的处理}
\TeX82的设定中,一般来说,行尾的carriage return做一个空格插入文本中处理。但
在汉字排版实践中,会造成多余的空隙,所以\JTeX 和\pTeX 都做了相应的额外处理。
\PUTeX 以及\hologo{XeTeX}都没有对这些行尾进行操作。而在\hologo{LuaTeX}中可以使用
Lua的callbacks进行干预。

\begin{itemize}
\item \JTeX 引入了\verb!\jendlinetype!进行处理,该命令为整数赋值型,可选值为:
\textit{jend\_ascii} (0),\textit{jend\_comment\_bit} (1),
\textit{jend\_ignore\_bit} (2)。

\item \pTeX 则没有如\JTeX 这种处理方式,“汉字+CR”中的CR不会处理为空格,但
“ASCII+CR”中的CR会被处理为空格。但,\pTeX 将\TeX82的状态处理扩展为:
\textit{mid\_line},\textit{mid\_kanji},\textit{skip\_blanks},\textit{new\_line}。
这样使\pTeX 多了数十个状态来控制汉字与非汉字的处理。
\end{itemize}
\end{frame}
%
\begin{frame}[fragile]
\frametitle{字体的处理}
\TeX82的设定中,字体,亦即font指的实际上是metrics,也可以说是通常认知上的font的抽象版本。
这其实和\TeX 的算法相关,即只需要宽度高度深度可以进行断行算法等的处理。
\begin{itemize}
\item \JTeX 中使用subfont技术,也就是将一个外部的大字体分割为多个小字体。这样做的好处是
每个码位都会有非常详细的信息。
\item \pTeX 中使用的是char class技术,是用了扩展的TFM格式,即JFM格式。JFM技术的优点是可以
将外部字体的所有信息归类分为几个大的class。但,JFM格式实际上可以覆盖整个Unicode区域,在
后期将JFM映射到外部字体的时候,需要比较复杂的VF配置。
\item \PUTeX 中使用的是fake font技术,实现上最为简单,但其通过一些额外的primitive进行了
对于一部分外部Virtual Font的处理。\PUTeX 会将字体信息最终存储到扩展了的DVI文件CDI文件中。
实际上这部分处理和\hologo{XeTeX}的比较相似。

\end{itemize}
\end{frame}
%
% http://troydhanson.github.io/tpl/
\begin{frame}[fragile]
\frametitle{\TeX 中的format或者数据序列化}
在\TeX 中,为了快速加载宏以及字体信息,通常需要读取format文件,扩展名为
\texttt{.fmt}。比如我们常见的\texttt{latex.fmt}文件就就是含有整套的\LaTeX 宏和
相关字体的。\TeX 中将内部数据结构存到文件的过程叫做dump,用现在的话来说,叫做
serialization。

在现代操作系统以及现代的\TeX 扩展当中,format文件已经有了不少的变化。比如:
\begin{itemize}
\item 压缩,主要通过zlib进行压缩,其主要的目的是减少C底层的IO操作
\item sparse化,涉及到内部的一些sparse array处理,主要是为了减少文件大小
\item endian控制,现在的计算机中的endian基本上有Big-Endian和Little-Endian两种,
这涉及到生成的format文件的跨平台行为
\end{itemize}
\end{frame}
%
\begin{frame}[fragile]
\frametitle{\ptexng 的catcode}
catcode是\TeX 及其扩展的一个核心概念,即应用于一套编码中所有码点(code point)
的一个属性值。亦即,可构造出一个带有附加属性的编码表。\TeX82、\hologo{XeTeX}和
\hologo{LuaTeX}的catcode有16种,而\ptexng 则有20种,多出的4种主要用来处理汉字。
在具体使用中,catcode通常用来动态改变码点的语义,最终会影响到\TeX 内部的状态机
处理。

\begin{center}
\newcommand{\pcc}[1]{\textcolor{red}{#1}}
\begin{tabular}{clcl}
0 &escape character    &10       &space\\
1 &beginning of group  &11       &letter\\
2 &end of group        &12       &other character\\
3 &math shift          &13       &active charactive\\
4 &alignment tab       &14       &comment character\\
5 &end of line         &15       &invalid character\\
6 &parameter           &\pcc{16} &\pcc{kanji}\\
7 &superscript         &\pcc{17} &\pcc{hiragana, katakana, alphabet}\\
8 &subscript           &\pcc{18} &\pcc{cjk symbol codes}\\
9 &ignored charcacter  &\pcc{19} &\pcc{hangul codes}\\
\end{tabular}
\end{center}
\end{frame}
%
\begin{frame}[fragile]
\frametitle{\ptexng 的node/noad}
node/noad是\TeX 在读取文本后,根据文本内容生成的链表元素。\TeX 核心的功能就是生成链表,
并根据链表最后输出DVI或者PDF。\ptexng 中新添加了两种新node,一个是dir,用来控制输出
内容链表的方向;另一个是disp,用来控制汉字字串基线的偏移。

\begin{center}
\newcommand{\pcc}[1]{\textcolor{red}{#1}}
\begin{tabular}{clclclcl}
0      &hlist      &10 &whatsit &20 &bin     &30 &left\\
1      &vlist      &11 &math    &21 &rel     &31 &right\\
\pcc{2}&\pcc{dir}  &12 &glue    &22 &open    &32 &---\\
3      &rule       &13 &kern    &23 &close   &33 &---\\
4      &ins        &14 &penalty &24 &punct   &34 &---\\
\pcc{5}&\pcc{disp} &15 &unset   &25 &inner   &35 &---\\
6      &mark       &16 &style   &26 &radical &36 &---\\
7      &adjust     &17 &choice  &27 &under   &37 &---\\
8      &ligature   &18 &ord     &28 &over    &38 &---\\
9      &disc       &19 &op      &29 &accent  &39 &---\\
\end{tabular}
\end{center}
\end{frame}
%
% Font Layout Table: Knowlege base for Complex Text Layout
%   http://www.unifont.org/TextLayout2007/presentations/KenichiHandaFLT.pdf
%
% FLT: Font Layout Table
%   http://repository.kulib.kyoto-u.ac.jp/dspace/bitstream/2433/65873/31/052-handa.pdf
%   http://coe21.zinbun.kyoto-u.ac.jp/papers/ws-type-2003/052-handa.pdf
%
%\begin{frame}[fragile]
%\frametitle{\ptexng 之multilingualization}
%\begin{center}
%\includegraphics[bb=0 0 106 118, width=6zw]{parrot.png}
%\end{center}
%\end{frame}
%
\begin{frame}[fragile]
\frametitle{\ptexng 的Font Layout Engine}
%\ptexng 使用的Font Layout Engine是由Kenichi Handa(半田剣一)、
%Mikiko Nishikimi(錦見美貴子)、Naoto Takahashi(高橋直人)和
%Satoru Tomura(戸村哲)合作开发的m17n及libotf。
\ptexng 使用的Font Layout Engine是日本产业技术综合研究所的半田剑一、
锦见美贵子、高桥直人和户村哲等合作开发的\texttt{m17n}库提供。
\texttt{m17n}即multilingualization,其开发思想最早可以追溯到Emacs上
的mule多语言处理模块。

\texttt{m17n}的字体呈现主要是通过\texttt{libotf}及\texttt{m17n-flt}
实现的。和其它的字体处理引擎(ICU、HarfBuzz、Graphite2)相比,
\texttt{m17n-flt}最便利的一点是可以通过类lisp的FLT脚本(即
{\bf F}ont {\bf L}ayout {\bf T}able)进行配置。此外还有数个
预先定义好的fontset可供使用,用来处理单个字体无法覆盖整个
Unicode全区域的情况,即相当于某些操作系统下的font fallbacks。

\texttt{libm17n}所提供的字体相关功能包括:

\begin{itemize}
\item 管 理(font cache中查找,通过fontconfig)
\item 呈 现(rendering,即渲染成二维字串,字即glyph)
\item 字体集(fontset,源于Emacs及X11)
\end{itemize}
\end{frame}
%
\begin{frame}[fragile]
\frametitle{\TeX 中的font描述形式}
无论是在\hologo{XeTeX}还是在\hologo{LuaTeX}中都可以使用OpenType字体。
但在语法使用上需要使用下列形式(X为\hologo{XeTeX},L为\hologo{LuaTeX}):
\begin{itemize}
\item \texttt{X} \verb!\font\Hoefler = "Hoefler Text/BI"!
\item \texttt{L} \verb!\font\pagella = {name:TeX Gyre Pagella} at 9pt!
\item \texttt{L} \verb!\font\LMR     = LatinModernRoman:clig=true;kern=false!
\end{itemize}

实际上,无论是\hologo{XeTeX}还是\hologo{LuaTeX}的\verb!\font!语法都比较
复杂,其本质是OpenType的使用较为复杂。OpenType的使用涉及到三大方面:script,
language,features。而另一方面,OpenType的调用是另一个复杂的问题。按照\TeX 最初
的设计思路,我们可以首先使用文件名来调用,其次,我们可以使用字体内部的名字进行
调用(见Uniscribe、GDI和fontconfig)。
\end{frame}
%
\begin{frame}[fragile]
\frametitle{X11下的font描述形式}
在\texttt{m17n}中,font的使用方式相对固定,比如有预定义的字体操作,例如FLT。
通常而言,只需要在m17n的代码中使用字体的唯一标识名称即可。在描述上,\texttt{m17n}
支持XLFD({\bf X} {\bf L}ogical {\bf F}ont {\bf D}escription)。其形式如下:

\begin{itemize}
\item \verb!-microsoft-sylfaen-normal-r-normal-*-unicode-bmp!
\item \verb!-adobe-source code pro-light-r-normal-*-unicode-bmp!
\item \verb!-monotype-shonar bangla-bold-r-normal-*-unicode-bmp!
\item \verb!-itc-pristina-normal-r-normal-*-unicode-bmp!
\end{itemize}

对于\TeX 来说,使用这种格式可能会使\verb!\font!的命令变得冗长。但对\texttt{m17n}
来说,这种格式便于解析和便于抛出错误。且由于格式是唯一的,对于使用\texttt{dvipdfmx}
生成PDF之时的字体映射来说,更为方便。
\end{frame}
%
\begin{frame}[fragile]
\frametitle{\texttt{m17n}中对于OpenType的配置}
\texttt{m17n}支持的OpenType配置有下面的语法(另有兼容语法,此处不讲):
\begin{itemize}
\item \verb!:otf=guru=blwf!, \verb!:otf=deva=half+!
\item \verb!:otf=deva=abvs,blws,psts,haln+abvm,blwm,dist!
\end{itemize}
即为(类BNF语法):
\begin{itemize}
\item OTF-spec ::= \verb!':otf='! \textit{script} \textit{langsys} \verb!?!
\textit{GSUB-features} \verb!?! \textit{GPOS-features} \verb!?!
\item script ::= \textit{symbol}
\item langsys ::= \verb!'/'! \textit{symbol}
\item GSUB-features ::= \verb!'='! \textit{feature-list} \verb!?!
\item GPOS-features ::= \verb!'+'! \textit{feature-list} \verb!?!
\item feature-list ::= \verb!(! \textit{symbol} \verb!',' ) * [! \textit{symbol} \verb!| '*' ]!
\end{itemize}
\end{frame}
%
\begin{frame}[fragile]
\frametitle{\ptexng 的primitive}
\begin{itemize}
\item \verb!\jfont!和\verb!\tfont!,这两个和\verb!\font!语义一致。
\pTeX 中的TFM有两种,一种是\TeX82定义了的TFM格式;另一种是\pTeX 定义的
专门用于重载字符宽度的JFM(后文会专门介绍JFM)。在JFM中,已经定义了对应的
文字方向,所以无论使用上述三种那个都不会产生问题。另外,还支持级数和齿数,
即Q和H,在大小上为0.25cm,在传统意义上只有前者可用于定义字体大小,
后者用来指配其他非字体元素的尺寸。
\item \begin{verbatim}
\font\tenmin=upjisr-h % mincho(KANJI)
\font\sevenmin=upjisr-h at 7pt
\font\fivemin=upjisr-h at 5pt
\end{verbatim}
\end{itemize}
\end{frame}
%
\begin{frame}[fragile]
\frametitle{\ptexng 的primitive}
\begin{itemize}
\item \verb!\kanjiskip!和\verb!\xkanjiskip!,前者会被自动插入到
汉字之间,后者会自动插入到非汉字与汉字之间。在\TeX82上,可以使用
\texttt{ex}和\texttt{em}。而在\pTeX 中,可以使用\texttt{zw}和\texttt{zh},
z表示\textit{zenkaku}(全角,ぜんかく),前者为全角宽度,后者为全角高度。
\item \begin{verbatim}
\kanjiskip=0pt plus .4pt minus .4pt
%\xkanjiskip=2.5pt plus 1pt minus 1pt
\xkanjiskip=.25zw plus 1pt minus 1pt
\end{verbatim}
\end{itemize}
\end{frame}
%
\begin{frame}[fragile]
\frametitle{\ptexng 的primitive}
\begin{itemize}
\item \verb!\tbaselineshift!和\verb!\ybaselineshift!,控制基线的偏移值。
前者为直行文本的偏移量,请注意直行的汉字的基线位垂直方向的中心;后者为
横行文本的基线偏移量。
\item \begin{verbatim}
\tbaselineshift=3pt
\ybaselineshift=3pt
\end{verbatim}
\item \def\showshift#1{\ybaselineshift#1 \fboxsep.4pt\fbox{林}\hskip\xkanjiskip\fbox{hayasi}\hskip\xkanjiskip\fbox{林}}
\showshift{-1pt}, \showshift{2pt}
\end{itemize}
\end{frame}
%
% \quitvmode
\begin{frame}[fragile]
\frametitle{\ptexng 的primitive}
\begin{itemize}
\item \verb!\pdfstrcmp!,比较字符串,\LaTeX3必须
\item \verb!\quitvmode!,来自\textsc{pdf}\TeX
\item \verb!\ngbanner!,输出banner: \ngbanner
\item \verb!\ngostype!,输出当前系统名称,当前为:\ngostype
\item \verb!\tracingfontloaders!,显示字体载入信息
\item \verb!\pdfcompresslevel!,设定PDF文件压缩比率(0—9,当前为{\the\pdfcompresslevel})
\item \verb!\pdfminorversion!,设定PDF文件版本号(0—7,当前为{\the\pdfminorversion})
\item \verb!\synctex!,Sync\TeX 支持(当前为{\the\synctex})
\end{itemize}
\end{frame}
%
%\begin{frame}[fragile]
%\frametitle{特别感谢}
%\begin{center}
%\begin{tabular}{ccccccl}
%罗心澄&苏 杰&齐 亮&刘海洋&高 虎&孙 亮&徐腾飞\\
%夏晓昊&黄晨成&黄金泽&梁 海&孙 伟&都龙山&王侠兵\\
%李国宝&胡玉进&杨海宇&周 浩&曹梦迪&李任之&卢 燚\\
%吕文博&翟 羽&钟 毓&肖智博&罗希睿&李 清&张 康\\
%王 政&戴唯思&李天池&乔崇智&林 坤&朱文俊&张 麒\\
%王昭礼&常金龙&冷俊园&高学远&张伟文&杨笑生&张 潇\\
%王 昊&林 科&朱焕杰&李润泽&王 璐&杨林恒&陈甫鸼\\
%孙文全&荣健欣&汤进伟&唐俊荣&郝 运&罗晨星&朱 翀\\
%何春奇&林懿伦&刘思江&金晨羽&陈 浩&刘中阳&王 冠\\
%杨文清&贾泊崴&王 昊&朱 宁&朱佳宁&陈 欣&冷 轩\\
%鲁尚文&韩勇杰&张逸轩&王俊淇&张天晓&董 帅&王孟昌\\
%柳青林&李文涛&凌嘉鸿&李晓东&&&\\
%\end{tabular}
%\end{center}
%\end{frame}
\end{document}
