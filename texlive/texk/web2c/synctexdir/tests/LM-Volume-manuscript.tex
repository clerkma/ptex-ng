\documentclass[ a4paper, oneside]{amsart}

%\listfiles


\RequirePackage{amsmath}
\RequirePackage{bm}
\RequirePackage{amssymb}
\RequirePackage{upref}
\RequirePackage{amsthm}
\RequirePackage{enumerate}
%\RequirePackage{pb-diagram}
\RequirePackage{amsfonts}
\RequirePackage[mathscr]{eucal}
\RequirePackage{verbatim}
\RequirePackage{xr}


\def\@thm#1#2#3{%
  \ifhmode\unskip\unskip\par\fi
  \normalfont
  \trivlist
  \let\thmheadnl\relax
  \let\thm@swap\@gobble
  \let\thm@indent\indent % no indent
  \thm@headfont{\scshape}% heading font bold
  %\thm@notefont{\fontseries\mddefault\upshape}%
  \thm@notefont{}%
  \thm@headpunct{.}% add period after heading
  \thm@headsep 5\p@ plus\p@ minus\p@\relax
  \thm@preskip\topsep
  \thm@postskip\thm@preskip
  #1% style overrides
  \@topsep \thm@preskip               % used by thm head
  \@topsepadd \thm@postskip           % used by \@endparenv
  \def\@tempa{#2}\ifx\@empty\@tempa
    \def\@tempa{\@oparg{\@begintheorem{#3}{}}[]}%
  \else
    \refstepcounter{#2}%
    \def\@tempa{\@oparg{\@begintheorem{#3}{\csname the#2\endcsname}}[]}%
  \fi
  \@tempa
}




%Redefined commands


%Greek Letters

\newcommand{\al}{\alpha}
\newcommand{\bet}{\beta}
\newcommand{\ga}{\gamma}
\newcommand{\de}{\delta }
\newcommand{\e}{\epsilon}
\newcommand{\ve}{\varepsilon}
\newcommand{\f}{\varphi}
\newcommand{\h}{\eta}
\newcommand{\io}{\iota}
\newcommand{\tht}{\theta}
\newcommand{\ka}{\kappa}
\newcommand{\lam}{\lambda}
\newcommand{\m}{\mu}
\newcommand{\n}{\nu}
\newcommand{\om}{\omega}
\newcommand{\p}{\pi}
\newcommand{\vt}{\vartheta}
\newcommand{\vr}{\varrho}
\newcommand{\s}{\sigma}
\newcommand{\x}{\xi}
\newcommand{\z}{\zeta}

\newcommand{\C}{\varGamma}
\newcommand{\D}{\varDelta}
\newcommand{\F}{\varPhi}
\newcommand{\Lam}{\varLambda}
\newcommand{\Om}{\varOmega}
\newcommand{\vPsi}{\varPsi}
\newcommand{\Si}{\varSigma}

%New Commands

\newcommand{\di}[1]{#1\nobreakdash-\hspace{0pt}dimensional}%\di n
\newcommand{\nbdd}{\nobreakdash--}
\newcommand{\nbd}{\nobreakdash-\hspace{0pt}}
\newcommand{\ce}[1]{$C^#1$\nbd{estimate}}
\newcommand{\ces}[1]{$C^#1$\nbd{estimates}}


\newcommand{\fm}[1]{F_{|_{M_#1}}}
\newcommand{\fmo}[1]{F_{|_{#1}}}%\fmo M
\newcommand{\fu}[3]{#1\hspace{0pt}_{|_{#2_#3}}}
\newcommand{\fv}[2]{#1\hspace{0pt}_{|_{#2}}}
\newcommand{\cchi}[1]{\chi\hspace{0pt}_{_{#1}}}
\newcommand{\so}{{\mc S_0}}
%\newcommand\sql[1][u]{\sqrt{1-|D#1|^2}}


\newcommand{\const}{\tup{const}}


\newcommand{\slim}[2]{\lim_{\substack{#1\ra #2\\#1\ne #2}}}


\newcommand{\pih}{\frac{\pi}{2}}


\newcommand{\msp[1]}[1]{\mspace{#1mu}}
\newcommand{\low}[1]{{\hbox{}_{#1}}}



%Special Symbols

\newcommand{\R}[1][n+1]{{\protect\mathbb R}^{#1}}
\newcommand{\Cc}{{\protect\mathbb C}}
\newcommand{\K}{{\protect\mathbb K}}
\newcommand{\N}{{\protect\mathbb N}}
\newcommand{\Q}{{\protect\mathbb Q}}
\newcommand{\Z}{{\protect\mathbb Z}}
\newcommand{\eR}{\stackrel{\lower1ex \hbox{\rule{6.5pt}{0.5pt}}}{\msp[3]\R[]}}
\newcommand{\eN}{\stackrel{\lower1ex \hbox{\rule{6.5pt}{0.5pt}}}{\msp[1]\N}}
\newcommand{\eO}{\stackrel{\lower1ex
\hbox{\rule{6pt}{0.5pt}}}{\msc O}}




%Special math symbols

\DeclareMathOperator{\arccot}{arccot}
\DeclareMathOperator{\diam}{diam}
\DeclareMathOperator{\Grad}{Grad}
\DeclareMathOperator*{\es}{ess\,sup}
\DeclareMathOperator{\graph}{graph}
\DeclareMathOperator{\sub}{sub}
\DeclareMathOperator{\supp}{supp}
\DeclareMathOperator{\id}{id}
\DeclareMathOperator{\lc}{lc}
\DeclareMathOperator{\osc}{osc}
\DeclareMathOperator{\pr}{pr}
\DeclareMathOperator{\rec}{Re}
\DeclareMathOperator{\imc}{Im}
\DeclareMathOperator{\sign}{sign}
\DeclareMathOperator{\proj}{proj}
\DeclareMathOperator{\grad}{grad}
\DeclareMathOperator{\Diff}{Diff}
\DeclareMathOperator{\rg}{rg}


\newcommand\im{\implies}
\newcommand\ra{\rightarrow}
\newcommand\xra{\xrightarrow}
\newcommand\rra{\rightrightarrows}
\newcommand\hra{\hookrightarrow}
\newcommand{\nea}{\nearrow}
\newcommand{\sea}{\searrow}
\newcommand{\ua}{\uparrow}
\newcommand{\da}{\downarrow}
\newcommand{\rha}{\rightharpoondown}
\newcommand{\wha}{\underset{w^*}\rightharpoondown}

%PDE commands

\newcommand\pa{\partial}
\newcommand\pde[2]{\frac {\partial#1}{\partial#2}}
\newcommand\pd[3]{\frac {\partial#1}{\partial#2^#3}}   %e.g. \pd fxi
\newcommand\pdc[3]{\frac {\partial#1}{\partial#2_#3}}   %contravariant
\newcommand\pdm[4]{\frac {\partial#1}{\partial#2_#3^#4}}   %mixed
\newcommand\pdd[4]{\frac {{\partial\hskip0.15em}^2#1}{\partial {#2^
#3}\,\partial{#2^#4}}}    %e.g. \pdd fxij, Abl. zweiter Ordnung
\newcommand\pddc[4]{\frac {{\partial\hskip0.15em}^2#1}{\partial {#2_
#3}\,\partial{#2_#4}}} 
\newcommand\PD[3]{\frac {{\partial\hskip0.15em}^2#1}{\partial
#2\,\partial#3}}       %e.g \PD fxy

\newcommand\df[2]{\frac {d#1}{d#2}}


\newcommand\sd{\vartriangle}
\newcommand\sq[1][u]{\sqrt{1+|D#1|^2}}
\newcommand\sql[1][u]{\sqrt{1-|D#1|^2}}
\newcommand{\un}{\infty}
\newcommand{\A}{\forall}
\newcommand{\E}{\exists}

%Set commands

\newcommand{\set}[2]{\{\,#1\colon #2\,\}}
\newcommand{\uu}{\cup}
\newcommand{\ii}{\cap}
\newcommand{\uuu}{\bigcup}
\newcommand{\iii}{\bigcap}
\newcommand{\uud}{ \stackrel{\lower 1ex \hbox {.}}{\uu}}
\newcommand{\uuud}[1]{ \stackrel{\lower 1ex \hbox {.}}{\uuu_{#1}}}
\newcommand\su{\subset}
\newcommand\Su{\Subset}
\newcommand\nsu{\nsubset}
\newcommand\eS{\emptyset}
\newcommand{\sminus}[1][28]{\raise 0.#1ex\hbox{$\scriptstyle\setminus$}}
\newcommand{\cpl}{\complement}

\newcommand\inn[1]{{\stackrel{\msp[9]\circ}{#1}}}



%Embellishments

\newcommand{\ol}{\overline}
\newcommand{\pri}[1]{#1^\prime}
\newcommand{\whn}[1]{\widehat{(#1_n)}}
\newcommand{\wh}{\widehat}


%Logical commands

\newcommand{\wed}{\wedge}
\newcommand{\eqv}{\Longleftrightarrow}
\newcommand{\lla}{\Longleftarrow}
\newcommand{\lra}{\Longrightarrow}
\newcommand{\bv}{\bigvee}
\newcommand{\bw}{\bigwedge}

\newcommand{\nim}{{\hskip2.2ex\not\hskip-1.5ex\im}}

\DeclareMathOperator*{\Au}{\A}
\DeclareMathOperator*{\Eu}{\E}

\newcommand\ti{\times }


%Norms
\newcommand{\abs}[1]{\lvert#1\rvert}
\newcommand{\absb}[1]{\Bigl|#1\Bigr|}
\newcommand{\norm}[1]{\lVert#1\rVert}
\newcommand{\normb}[1]{\Big\lVert#1\Big\rVert}
\newcommand{\nnorm}[1]{| \mspace{-2mu} |\mspace{-2mu}|#1| \mspace{-2mu}
|\mspace{-2mu}|}
\newcommand{\spd}[2]{\protect\langle #1,#2\protect\rangle}

%Geometry
\newcommand\ch[3]{\varGamma_{#1#2}^#3}
\newcommand\cha[3]{{\bar\varGamma}_{#1#2}^#3}
\newcommand{\riem}[4]{R_{#1#2#3#4}}
\newcommand{\riema}[4]{{\bar R}_{#1#2#3#4}}
\newcommand{\cod}{h_{ij;k}-h_{ik;j}=\riema\al\bet\ga\de\n^\al x_i^\bet x_j^\ga x_k^\de}
\newcommand{\gau}[1][\s]{\riem ijkl=#1 \{h_{ik}h_{jl}-h_{il}h_{jk}\} + \riema
\al\bet\ga\de x_i^\al x_j^\bet x_k^\ga x_l^\de}
\newcommand{\ric}{\h_{i;jk}=\h_{i;kj}+\riem lijk\msp \h^l}

%Font commands

\newcommand{\tbf}{\textbf}
\newcommand{\tit}{\textit}
\newcommand{\tsl}{\textsl}

\newcommand{\tsc}{\textsc}
\newcommand{\trm}{\textrm}
\newcommand{\tup}{\textup}% text upright

\newcommand{\mbf}{\protect\mathbf}
\newcommand{\mitc}{\protect\mathit}
\newcommand{\mrm}{\protect\mathrm}


\newcommand{\bs}{\protect\boldsymbol}
\newcommand{\mc}{\protect\mathcal}
\newcommand{\msc}{\protect\mathscr}



%Miscellaneous

\providecommand{\bysame}{\makeboc[3em]{\hrulefill}\thinspace}
\newcommand{\la}{\label}
\newcommand{\ci}{\cite}
\newcommand{\bib}{\bibitem}

\newcommand{\cq}[1]{\glqq{#1}\grqq\,}
\newcommand{\cqr}{\glqq{$\lra$}\grqq\,}
\newcommand{\cql}{\glqq{$\lla$}\grqq\,}

\newcommand{\bt}{\begin{thm}}
\newcommand{\bl}{\begin{lem}}
\newcommand{\bc}{\begin{cor}}
\newcommand{\bd}{\begin{definition}}
\newcommand{\bpp}{\begin{prop}}
\newcommand{\br}{\begin{rem}}
\newcommand{\bn}{\begin{note}}
\newcommand{\be}{\begin{ex}}
\newcommand{\bes}{\begin{exs}}
\newcommand{\bb}{\begin{example}}
\newcommand{\bbs}{\begin{examples}}
\newcommand{\ba}{\begin{axiom}}



\newcommand{\et}{\end{thm}}
\newcommand{\el}{\end{lem}}
\newcommand{\ec}{\end{cor}}
\newcommand{\ed}{\end{definition}}
\newcommand{\epp}{\end{prop}}
\newcommand{\er}{\end{rem}}
\newcommand{\en}{\end{note}}
\newcommand{\ee}{\end{ex}}
\newcommand{\ees}{\end{exs}}
\newcommand{\eb}{\end{example}}
\newcommand{\ebs}{\end{examples}}
\newcommand{\ea}{\end{axiom}}


\newcommand{\bp}{\begin{proof}}
\newcommand{\ep}{\end{proof}}
\newcommand{\eps}{\renewcommand{\qed}{}\end{proof}}

\newcommand{\bal}{\begin{align}}
%\newcommand{\eal}{\end{align}}


\newcommand{\bi}[1][1.]{\begin{enumerate}[\upshape #1]}
\newcommand{\bia}[1][(1)]{\begin{enumerate}[\upshape #1]}
\newcommand{\bin}[1][1]{\begin{enumerate}[\upshape\bfseries #1]}
\newcommand{\bir}[1][(i)]{\begin{enumerate}[\upshape #1]}
\newcommand{\bic}[1][(i)]{\begin{enumerate}[\upshape\hspace{2\cma}#1]}
\newcommand{\bis}[2][1.]{\begin{enumerate}[\upshape\hspace{#2\parindent}#1]}
\newcommand{\ei}{\end{enumerate}}



% comma is raised when components are quotients

\newcommand\ndots{\raise 0.47ex \hbox {,}\hskip0.06em\cdots %
     \raise 0.47ex \hbox {,}\hskip0.06em} 

%Layout commands


\newcommand{\clearemptydoublepage}{\newpage{\pagestyle{empty}\cleardoublepage}}
\newcommand{\q}{\quad}
\newcommand{\qq}{\qquad}

\newcommand{\vs}[1][3]{\vskip#1pt}
\newcommand{\hs}[1][12]{\hskip#1pt}

\newcommand{\hp}{\hphantom}
\newcommand{\vp}{\vphantom}

\newcommand\cl{\centerline}

\newcommand\nl{\newline}

\newcommand\nd{\noindent}

\newcommand{\nt}{\notag}

% %my private skips; set to 0 to restore default

\newskip\Csmallskipamount                                                
\Csmallskipamount=\smallskipamount
\newskip\Cmedskipamount
\Cmedskipamount=\medskipamount
\newskip\Cbigskipamount
\Cbigskipamount=\bigskipamount

\newcommand\cvs{\vspace\Csmallskipamount}   
\newcommand\cvm{\vspace\Cmedskipamount}
\newcommand\cvb{\vspace\Cbigskipamount}


\newskip\csa
\csa=\smallskipamount

\newskip\cma
\cma=\medskipamount

\newskip\cba
\cba=\bigskipamount

\newdimen\spt
\spt=0.5pt


%%special roster macro

\newcommand\citem{\cvs\advance\itemno by
1{(\romannumeral\the\itemno})\hskip3pt}
\newcommand{\bitem}{\cvm\nd\advance\itemno by
1{\bf\the\itemno}\hspace{\cma}}
\newcommand\cendroster{\cvm\itemno=0}


%New counts

\newcount\itemno
\itemno=0

%Labels

\newcommand{\las}[1]{\label{S:#1}}
\newcommand{\lass}[1]{\label{SS:#1}}
\newcommand{\lae}[1]{\label{E:#1}}
\newcommand{\lat}[1]{\label{T:#1}}
\newcommand{\lal}[1]{\label{L:#1}}
\newcommand{\lad}[1]{\label{D:#1}}
\newcommand{\lac}[1]{\label{C:#1}}
\newcommand{\lan}[1]{\label{N:#1}}
\newcommand{\lap}[1]{\label{P:#1}}
\newcommand{\lar}[1]{\label{R:#1}}
\newcommand{\laa}[1]{\label{A:#1}}

%Referencing

\newcommand{\rs}[1]{Section~\ref{S:#1}}
\newcommand{\rss}[1]{Section~\ref{SS:#1}}
\newcommand{\rt}[1]{Theorem~\ref{T:#1}}
\newcommand{\rl}[1]{Lemma~\ref{L:#1}}
\newcommand{\rd}[1]{Definition~\ref{D:#1}}
\newcommand{\rc}[1]{Corollary~\ref{C:#1}}
\newcommand{\rn}[1]{Number~\ref{N:#1}}
\newcommand{\rp}[1]{Proposition~\ref{P:#1}}
\newcommand{\rr}[1]{Remark~\ref{R:#1}}
\newcommand{\raa}[1]{Axiom~\ref{A:#1}}
\newcommand{\re}[1]{\eqref{E:#1}}


%Index
\newcommand{\ind}[1]{#1\index{#1}}






\RequirePackage{upref}
\RequirePackage{amsthm}
%\usepackage{amsfonts}
%\usepackage{amsintx}
\RequirePackage{enumerate}%\begin{enumerate}[(i)]

%%\usepackage{showkeys}
\setlength{\textwidth}{4.7in}%JDG
\setlength{\textheight}{7.5in}

\usepackage{germanquotes}

\theoremstyle{plain}
\newtheorem{thm}{Theorem}[section]
\newtheorem{lem}[thm]{Lemma}
\newtheorem{prop}[thm]{Proposition}
\newtheorem{cor}[thm]{Corollary}

\theoremstyle{definition}
\newtheorem{rem}[thm]{Remark}
\newtheorem{definition}[thm]{Definition}
\newtheorem{example}[thm]{Example}
\newtheorem{ex}[thm]{Exercise}

\swapnumbers
\theoremstyle{remark}
\newtheorem{case}{Case}

\numberwithin{equation}{section}

%\renewcommand{\qed}{q.e.d.}

\usepackage{xr-hyper}
\usepackage{url}
\usepackage[hyperindex=true, pdfauthor= Claus\  Gerhardt, pdftitle= LM-Volume, bookmarks=true, extension= pdf, colorlinks=true, plainpages=false,hyperfootnotes=true, debug=false, pagebackref]{hyperref}

\newcommand{\anl}{\htmladdnormallink}

%\listfiles
\begin{document}
%\larger[1]
\title{Estimates for the volume of a Lorentzian manifold}

% author one information
\author{Claus Gerhardt}
\address{Ruprecht-Karls-Universit\"at, Institut f\"ur Angewandte Mathematik,
Im Neuenheimer Feld 294, 69120 Heidelberg, Germany}
%\curraddr{}
\email{gerhardt@math.uni-heidelberg.de}
\urladdr{\url{http://www.math.uni-heidelberg.de/studinfo/gerhardt/}}
%\thanks{}

% author two information
%\author{}
%\address{}
%\curraddr{}
%\email{}
%\thanks{}
%
\subjclass[2000]{35J60, 53C21, 53C44, 53C50, 58J05}
\keywords{Lorentzian manifold, volume estimates, cosmological spacetime, general relativity, constant mean curvature, CMC hypersurface}
\date{April 18, 2002}
%
% at present the "communicated by" line appears only in ERA and PROC
%\commby{}

%\dedicatory{}

\begin{abstract} We prove new estimates for the volume of a Lorentzian
mani\-fold and show especially that cosmological spacetimes with crushing
singularities have finite volume.
\end{abstract}
\maketitle
\thispagestyle{empty}

\setcounter{section}{-1}
\section{Introduction} 

\cvb
Let $N$ be a  $(n+1)$-dimensional Lorentzian manifold and suppose that $N$ can be
decomposed in the form

\begin{equation}\lae{0.1}
N=N_0\uu N_-\uu N_+,
\end{equation}

\cvm 
\nd where $N_0$ has finite volume and $N_-$ resp. $N_+$ represent the critical
past resp. future Cauchy developments with not necessarily a priori bounded
volume. We assume that $N_+$ is the future Cauchy development of a Cauchy
hypersurface $M_1$, and $N_-$ the past Cauchy development of a hypersurface
$M_2$, or, more precisely, we assume the existence of a time function $x^0$,
such that

\begin{equation}
\begin{aligned}
N_+&={x^0}^{-1}([t_1,T_+)),&\qq M_1=\{x^0=t_1\}&,\\
N_-&={x^0}^{-1}((T_-,t_2]),&\qq M_2=\{x^0=t_2\}&,
\end{aligned}
\end{equation}

\cvm
\nd and that the Lorentz metric can be expressed as

\begin{equation}\lae{0.3}
d\bar s^2=e^{2\psi}\{-{dx^0}^2+\s_{ij}(x^0,x)dx^idx^j\},
\end{equation}

\cvm
\nd where $x=(x^i)$ are local coordinates for the space-like hypersurface $M_1$
if $N_+$ is considered resp. $M_2$ in case of $N_-$.

The coordinate system $(x^\al)_{0\le\al\le n}$ is supposed to be future
directed, i.e. the \tit{past} directed unit normal $(\nu^\al)$ of the level sets

\begin{equation}
M(t)=\{x^0=t\}
\end{equation}

\cvm
\nd is of the form

\begin{equation}\lae{0.5}
(\nu^\al)=-e^{-\psi}(1,0,\ldots,0).
\end{equation}

\cvm
If we assume the mean curvature of the slices $M(t)$ with respect to the past
directed normal---cf. \ci[Section 2]{cg8} for a more detailed explanation of our
conventions---is strictly bounded away from zero, then, the following volume
estimates can be proved

\bt\lat{0.1}
Suppose there exists a positive constant $\e_0$ such that


\begin{align}
H(t)&\ge \e_0&\A\,t_1\le t< T_+&,\lae{0.6}\\
\intertext{and}
H(t)&\le-\e_0&\A\,T_-<t\le t_2&,\lae{0.7}
\end{align}

\cvm
\nd then

\begin{align}
\abs{N_+}&\le \frac1{\e_0}\abs{M(t_1)},\\
\intertext{and}
\abs{N_-}&\le \frac1{\e_0}\abs{M(t_2}.
\end{align}

These estimates also hold locally, i.e. if $E_i\su M(t_i)$, $i=1,2$, are measurable
subsets and $E_1^+,E_2^-$ the corresponding future resp. past directed
cylinders, then,

\begin{align}
\abs{E_1^+}&\le\frac1{\e_0}\abs{E_1},\lae{0.10}\\
\intertext{and}
\abs{E_2^-}&\le\frac1{\e_0}\abs{E_2}.
\end{align}
\et

\cvb
\section{Proof of \rt{0.1}}\las{1}

\cvb
In the following we shall only prove the estimate for $N_+$, since the other case
$N_-$ can easily be considered as a future development by reversing the time
direction.

\cvm
Let $x=x(\xi)$ be an embedding of a space-like hypersurface and $(\nu^\al)$ be
the past directed normal. Then, we have the Gau{\ss} formula

\begin{equation}
x^\al_{ij}=h_{ij}\nu^\al.
\end{equation}

\cvm
\nd where $(h_{ij})$ is the second fundamental form, and the Weingarten equation

\begin{equation}
\nu^\al_i=h^k_ix^\al_k.
\end{equation}


\cvm
We emphasize that covariant derivatives, indicated simply by indices, are
always \tit{full} tensors.

\cvm
The slices $M(t)$ can be viewed as  special embeddings of the form

\begin{equation}
x(t)=(t,x^i),
\end{equation}

\cvm
\nd where $(x^i)$ are coordinates of the \tit{initial} slice $M(t_1)$. Hence, the
slices $M(t)$ can be considered as the solution of the evolution problem

\begin{equation}\lae{1.4}
\dot x=-e^\psi \nu, \qq t_1\le t<T_+,
\end{equation}

\cvm
\nd with initial hypersurface $M(t_1)$, in view of \re{0.5}.

\cvm From the equation \re{1.4} we can immediately derive evolution equations
for the geometric quantities $g_{ij}, h_{ij}, \nu$, and $H=g^{ij}h_{ij}$ of $M(t)$, cf.
e.g.
\ci[Section 4]{cg4}, where the corresponding evolution equations are derived in
Riemannian space.

\cvm
For our purpose, we are only interested in the evolution equation for the metric,
and we deduce

\begin{equation}
\dot g_{ij}=\spd{\dot x_i}{x_j}+\spd{x_i}{\dot x_j}=- 2e^\psi h_{ij},
\end{equation}

\cvm
\nd in view of the Weingarten equation.

\cvm
Let $g=\det(g_{ij})$, then,

\begin{equation}\lae{1.6}
\dot g= g g^{ij}\dot g_{ij}=-2e^\psi H g,
\end{equation}

\cvm
\nd and thus, the volume of $M(t), \abs{M(t)}$, evolves according to

\begin{equation}\lae{1.7}
\frac d{dt}  \abs{M(t)}=\int_{M(t_1)}\frac d{dt}\sqrt g=-\int_{M(t)}e^\psi H,
\end{equation}

\cvm
\nd where we shall assume without loss of generality that $\abs{M(t_1}$ is finite,
otherwise, we replace $M(t_1)$ by an arbitrary measurable subset of $M(t_1)$
with finite volume.

\cvm
Now, let $T\in [t_1, T_+)$ be arbitrary and denote by $Q(t_1,T)$ the
cylinder

\begin{equation}\lae{1.8}
Q(t_1,T)=\set{(x^0,x)}{t_1\le x^0\le T},
\end{equation}

\cvm
\nd then,

\begin{equation}\lae{1.9}
\abs{Q(t_1,T)}=\int_{t_1}^T\int_Me^\psi,
\end{equation}

\cvm
\nd where we omit the volume elements, and where, $M=M(x^0)$.

\cvm
By assumption, the mean curvature $H$ of the slices is bounded from below by
$\e_0$, and we conclude further, with the help of \re{1.7},

\begin{equation}
\begin{aligned}
\abs{Q(t_1,T)}&\le\frac 1{\e_0} \int_{t_1}^T\int_Me^\psi H\\
&=\frac1{\e_0}\{\abs{M(t_1)}-\abs{M(T)}\}\\
&\le \frac1{\e_0}\abs{M(t_1)}.
\end{aligned}
\end{equation}


\cvm
Letting $T$ tend to $T_+$ gives the estimate for $\abs {N_+}$.

\cvm
To prove the estimate \re{0.10}, we simply replace $M(t_1)$ by $E_1$.

\cvb
If we relax the conditions \re{0.6} and \re{0.7} to include the case $\e_0=0$, a
volume estimate is still possible.

\cvm
\bt
If the assumptions of \rt{0.1} are valid with $\e_0=0$, and if in addition the
length of any future directed curve starting from $M(t_1)$ is bounded by a
constant $\ga_1$ and the length of any past directed curve starting from $M(t_2)$
is bounded by a constant $\ga_2$, then,
\begin{align}
\abs{N_+}&\le \ga_1\abs{M(t_1)}\\
\intertext{and}
\abs{N_-}&\le \ga_2\abs{M(t_2)}.
\end{align}
\et

\cvm
\bp
As before, we only consider the estimate for $N_+$.

\cvm
From \re{1.6} we infer that the volume element of the slices $M(t)$ is decreasing
in $t$, and hence,
\begin{equation}\lae{1.13}
\sqrt{g(t)}\le \sqrt{g(t_1)}\qq\A\,t_1\le t.
\end{equation}

\cvm
Furthermore, for fixed $x\in M(t_1)$ and $t>t_1$
\begin{equation}\lae{1.14}
\int_{t_1}^te^\psi\le \ga_1
\end{equation}
because the left-hand side is the length of the future directed curve
\begin{equation}
\ga(\tau)=(\tau,x)\qq t_1\le\tau\le t.
\end{equation}

\cvm
Let us now look at the cylinder $Q(t_1,T)$ as in \re{1.8} and \re{1.9}. We have
\begin{equation}
\begin{aligned}
\abs{Q(t_1,T)}&=\int_{t_1}^T\int_{M(t_1)}e^\psi\sqrt{g(t,x)}\le
\int_{t_1}^T\int_{M(t_1)}e^\psi\sqrt{g(t_1,x)}\\[\cma]
&\le \ga_1\int_{M(t_1)}\sqrt{g(t_1,x)}=\ga_1\abs{M(t_1)}
\end{aligned}
\end{equation}
by applying Fubini's theorem and the estimates \re{1.13} and \re{1.14}.
\ep

\cvb
\section{Cosmological spacetimes}\las{2}

\cvb
A cosmological spacetime is a globally hyperbolic Lorentzian manifold $N$ with
compact Cauchy hypersurface $\so$, that satisfies the timelike convergence
condition, i.e.

\begin{equation}
\bar R_{\al\bet}\nu^\al\nu^\bet\ge 0 \qq \A\,\spd\nu\nu=-1.
\end{equation}

\cvm
If there exist crushing singularities, see \ci{es} or \ci{cg1} for a definition, then,
we proved in
\ci{cg1} that
$N$ can be foliated by spacelike hypersurfaces $M(\tau)$ of constant mean
curvature $\tau$, $-\un<\tau<\un$,

\begin{equation}
N=\uuu_{0\ne\tau\in \R[]}M(\tau)\uu{\msc C}_0,
\end{equation}


\cvm
\nd where $\msc C_0$ consists either of a single maximal slice or of a whole
continuum of maximal slices in which case the metric is stationary in $\msc
C_0$. But in any case $\msc C_0$ is a compact subset of $N$.

\cvm
In the complement of $\msc C_0$ the mean curvature function $\tau$ is a regular
function with non-vanishing gradient that can be used as a new time function, cf.
\ci{cg6} for a simple proof.

\cvm
Thus, the Lorentz metric can be expressed in Gaussian coordinates $(x^\al)$ with
$x^0=\tau$ as in \re{0.3}. We choose arbitrary $\tau_2<0<\tau_1$ and de\-fine

\begin{equation}
\begin{aligned}
N_0&=\set{(\tau,x)}{\tau_2\le\tau \le \tau_1},\\
N_-&=\set{(\tau,x)}{-\un<\tau \le \tau_2},\\
N_+&=\set{(\tau,x)}{\tau_1\le \tau<\un}.
\end{aligned}
\end{equation}

\cvm
Then, $N_0$ is compact, and the volumes of $N_-, N_+$ can be estimated by

\begin{align}
\abs{N_+}&\le \frac1{\tau_1}\abs{M(\tau_1)},\\
\intertext{and}
\abs{N_-}&\le \frac1{\abs{\tau_2}}\abs{M(\tau_2)}.
\end{align}

\cvm
Hence, we have proved

\bt
A cosmological spacetime $N$ with crushing singularities has finite volume.
\et

\cvb
\br
Let $N$ be a spacetime with compact Cauchy hypersurface and suppose that a
subset
$N_-\su N$ is foliated by constant mean curvature slices $M(\tau)$ such that

\begin{equation}
N_-=\uuu_{0<\tau\le \tau_2}M(\tau)
\end{equation}

\cvm
\nd and suppose furthermore, that $x^0=\tau$ is a time function---which will be
the case if the timelike convergence condition is satisfied---so that the metric
can be represented in Gaussian coordinates $(x^\al)$ with $x^0=\tau$.

\cvm
Consider the cylinder $Q(\tau,\tau_2)=\{\tau\le x^0\le \tau_2\}$ for some
fixed $\tau$. Then, 

\begin{equation}
\abs{Q(\tau,\tau_2)}=\int_\tau^{\tau_2}\int_Me^\psi=\int_\tau
^{\tau_2}H^{-1}\int_MH e^\psi,
\end{equation}

\cvm
\nd and we obtain in view of \re{1.7}

\begin{equation}
\tau^{-1}_2\{\abs {M(\tau)}-\abs{M(\tau_2)}\}\le\abs{Q(\tau,\tau_2)},
\end{equation}

\cvm
\nd and conclude further

\begin{equation}
\lim_{\tau\ra 0}\msp[2]\abs{M(\tau)}\le \tau_2\abs{N_-}+\abs{M(\tau_2)},
\end{equation}

\nd i.e.

\begin{equation}
\lim_{\tau\ra 0}\msp[2]\abs{M(\tau)}=\un\im \abs{N_-}=\un.
\end{equation}
\er

\cvb
\section{The Riemannian case}

\cvb
Suppose that $N$ is a Riemannian manifold that is decomposed as in \re{0.1} with
metric


\begin{equation}
d\bar s^2=e^{2\psi}\{{dx^0}^2+\s_{ij}(x^0,x)dx^idx^j\}.
\end{equation}

\cvm
The Gau{\ss} formula and the Weingarten equation for a hypersurface now have
the form

\begin{align}
x^\al_{ij}&=-h_{ij}\nu^\al,\\
\intertext{and}
\nu^\al_i&=h^k_ix^\al_k.
\end{align}


\cvm
As default normal vector---if such a choice is possible---we choose the outward
normal, which, in case of the coordinate slices $M(t)=\{x^0=t\}$ is given by

\begin{equation}
(\nu^\al)=e^{-\psi}(1,0,\ldots,0).
\end{equation}


\cvm
Thus, the coordinate slices are solutions of the evolution problem

\begin{equation}
\dot x=e^\psi \nu,
\end{equation}

\cvm
\nd and, therefore,

\begin{equation}
\dot g_{ij}=2e^\psi h_{ij},
\end{equation}

\cvm
\nd i.e. we have the opposite sign compared to the Lorentzian case leading to

\begin{equation}
\frac d{dt}\abs{M(t)}=\int_Me^\psi H.
\end{equation}

\cvm
The arguments in \rs{1} now yield

\bt
\tup{(i)} Suppose there exists a positive constant $\e_0$ such that the mean
curvature $H(t)$ of the slices $M(t)$ is estimated by

\begin{align}
H(t)&\ge \e_0&\A\,t_1\le t< T_+&,\\
\intertext{and}
H(t)&\le-\e_0&\A\,T_-<t\le t_2&,
\end{align}

\cvm
\nd then

\begin{align}
\abs{N_+}&\le \frac1{\e_0}\lim_{t\ra T_+}\abs{M(t)},\\
\intertext{and}
\abs{N_-}&\le \frac1{\e_0}\lim_{t\ra T_-}\abs{M(t}.
\end{align}

\cvm
\tup{(ii)} On the other hand, if the mean curvature $H$ is negative in $N_+$ and
positive in $N_-$, then, we obtain the same estimates as \rt{0.1}, namely,

\begin{align}
\abs{N_+}&\le \frac1{\e_0}\abs{M(t_1)},\\
\intertext{and}
\abs{N_-}&\le \frac1{\e_0}\abs{M(t_2)}.
\end{align}
\et

\cvb

\begin{thebibliography}{99}
\bib{es}
D. Eardley \& L. Smarr, \emph{Time functions in numerical relativity: marginally
bound dust collapse}, Phys. Rev. D \tbf{19} (1979) 2239\nbdd2259.


\bib{cg1}
C. Gerhardt, \emph{H-surfaces in Lorentzian manifolds}, Commun. Math. Phys.
\tbf{89} (1983) 523\nbdd{553}.



\bib{cg4}
\bysame, \emph{Hypersurfaces of prescribed Weingarten curvature}, Math. Z.
\tbf{224} (1997) 167\nbdd{194}.
\url{http://www.math.uni-heidelberg.de/studinfo/gerhardt/MZ224,97.pdf}



\bib{cg6}
\bysame, \emph{On the foliation of space-time by constant mean curvature
hypersurfaces}, preprint,
\url{http://www.math.uni-heidelberg.de/studinfo/gerhardt/Foliation.pdf}


\bib{cg8}
\bysame, \emph{Hypersurfaces of
prescribed  curvature in Lorentzian manifolds}, Indiana Univ. Math. J. \tbf{49}
(2000) 1125\nbdd1153.
\url{http://www.math.uni-heidelberg.de/studinfo/gerhardt/GaussLorentz.pdf}]





\bib{HE}
S. W. Hawking \& G. F. R. Ellis, \emph{The large scale structure of space-time},
Cambridge University Press, Cambridge, 1973.



\end{thebibliography}
\end{document}


%------------------------------------------------------------------------------
% End of journal.top
%------------------------------------------------------------------------------
